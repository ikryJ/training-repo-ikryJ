\documentclass[report,nochapters]{aaltoseries}
\usepackage[latin1]{inputenc}
% hack to emulate non-odd+even page layout, leaves page numbers in original places though
\setlength{\oddsidemargin}{5mm} 
\setlength{\evensidemargin}{5mm}

% can't get to work
%\usepackage[RGB,Teknillinen korkeakoulu,Kirjasto,Otaniemi,Coated]{aaltologo}

% apparently \maketitle is not implemented in this document class. Therefore in plain text below
\title{On the KPI process
\author{Tuija Sonkkila}
\date{11.5.2012}


% start
\begin{document}
\AaltoLogoLarge{1}{!}{aaltoYellow}

On the KPI process

Tuija Sonkkila

10.5.2012

\tableofcontents 

\section{Files, database and tables}

All scripts listed below are located on the CSC Hippu server, on my sonkkila account, in the directory 
\begin{verbatim}
$USERAPPL/tools/aalto/prosessi
\end{verbatim}
and they are also shown in this report in verbatim (deleted from this version).  

\begin{description}
    \item[calc.pl] Perl script. Calculates citations counts of a given WoS list
    \item[dofinal.pl] Perl script. Creates and populates all tables with final NCSf scores
    \item[scores.pl] Perl script. Runs world reference values by subject
    \item[scorescrange] Table that stores non-self citations averages of the combinations of every Wos field, year and publication type
    \item[jfieldscorescrange] Table that stores non-self citation averages of combinations of ISSN, year, and publication type. We want to find out citation averages over all WoS fields that constitute a given journal. Journals have 1-n subject fields, so we want to calculate a total average between these fields
    \item[aaltocrange] Table that stores all non-self citations of the given (Aalto) item run, and their NCSf values, calculated against world reference values
    \item[aaltofrom2003to2007] Table that stores the final NCSf score by unit, in this particular time window
\end{description}


\section{World reference values by WoS subject}

Create table scorescrange.

The function of the table is to store non-self citation averages of the combinations of every WoS field, year and publication type.

The Perl script scores.pl inserts lines to table scorescrange. The script and table need updating whenever new years and year ranges are calculated.

The table scorescrange is then updated by sum and average fields. Note that these SQL commands could also be appended in the scores.pl but as for now, are run manually.

\section{World reference values across subjects that constitute a journal}

Create table jfieldscorescrange.

The function of this table is to collect non-self citation averages of combinations of ISSN, year, and publication type. The goal is to find out citation averages over all WoS fields that constitute a given journal. Journals have 1-n subject fields, and we want to calculate a total average between these fields. 

\section{Average citation scores}

Create table aaltocrange.

The function of this table is to store all non-self citations of the given Aalto item run, and their NCSf values, calculated against world reference values.

The table schema needs updating whenever new years and year ranges are calculated.

Prepare a CSV file of Aalto items. Example:

\begin{verbatim}
2007%ISI:000206301700007%SCI
2007%ISI:000206312800016%ELEC
2008%ISI:000207455000011%SCI
2010%WOS:000208455700017%SCI
2003%ISI:000207559200006%SCI
\end{verbatim}

The Perl script calc.pl calculates non-self citation counts of WoS items in the input file.

Update table aaltocrange by counting citation sums over year ranges, and normalizing the scores against the world reference values found in table jfieldscorescrange.

These SQL commands could also be included in the calc.pl script, but as for now, are run manually.

Updates can now also be run by joining to the alljfieldscorescrange (see above about the table).

\section{Final scores}

Creates and populates final NCSf values by time window.

\end{document}
